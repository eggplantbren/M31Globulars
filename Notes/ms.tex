\documentclass[a4paper, 12pt]{article}
\usepackage{amsmath}
\usepackage{amssymb}
\usepackage{dsfont}
\usepackage[left=1.5cm, right=1.5cm, bottom=2cm, top=2cm]{geometry}
\usepackage{graphicx}
\usepackage{hyperref}
\usepackage[utf8]{inputenc}
\usepackage{microtype}
\usepackage{natbib}
\newcommand{\given}{\,|\,}

\title{Notes}
\author{Brendon J. Brewer}
\date{}

\begin{document}
\maketitle

\abstract{}

% Need this after the abstract
\setlength{\parindent}{0pt}
\setlength{\parskip}{8pt}

\section*{Unknown True Metallicities}
Let $\{Z_i\}$ be the unknown true metallicities and let $\{z_i\}$ be the
measured metallicities, with the reported errorbars $\{b_i\}$, such that
\begin{align}
z_i \given Z_i &\sim \textnormal{Normal}\left(z_i, b_i^2\right).
\end{align}
Let the other parameters be $\theta$ and the data be $D$. We seek the
posterior
\begin{align}
p(\theta, \boldsymbol{Z} \given D) &\propto p(\theta, \boldsymbol{Z})
                                            p(D \given \theta, \boldsymbol{Z}) \\
    &= p(\theta)p(\boldsymbol{Z}) p(D \given \theta, \boldsymbol{Z}),
\end{align}
assuming that the true metallicities do not tell us anything about the other
parameters. Then, we can marginalise out the true metallicities:
\begin{align}
p(\theta \given D)
    &= \int p(\theta, \boldsymbol{Z} \given D) \, d\boldsymbol{Z} \\
    &\propto \int p(\theta)p(\boldsymbol{Z}) p(D \given \theta, \boldsymbol{Z})
               \, d\boldsymbol{Z} 
\end{align}


\end{document}

